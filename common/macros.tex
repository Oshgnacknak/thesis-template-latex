\tikzset{
    sffont/.style={font=\sffamily},
    base/.style={
        sffont,
        thick,
        rounded corners=2pt,
    },
    scnode/.style={
        base,
        rectangle,
        draw,
        fill=\thepagecolor,
        align=center,
        minimum height=1cm,
        minimum width=1cm,
    },
    legend/.style={fill=fgcolor!\IfDarkModeTF{10}{5}!\thepagecolor,anchor=north east,align=left,rounded corners=2pt},
}

\makeatletter
% taken from pgf manual
\pgfdeclareshape{document}{
    \inheritsavedanchors[from=rectangle] % this is nearly a rectangle
    \inheritanchorborder[from=rectangle]
    \inheritanchor[from=rectangle]{center}
    \inheritanchor[from=rectangle]{north}
    \inheritanchor[from=rectangle]{south}
    \inheritanchor[from=rectangle]{west}
    \inheritanchor[from=rectangle]{east}
    % ... and possibly more
    \backgroundpath{% this is new
        % store lower right in xa/ya and upper right in xb/yb
        \southwest \pgf@xa=\pgf@x \pgf@ya=\pgf@y
        \northeast \pgf@xb=\pgf@x \pgf@yb=\pgf@y
        % compute corner of ``flipped page''
        \pgf@xc=\pgf@xb \advance\pgf@xc by-5pt % this should be a parameter
        \pgf@yc=\pgf@yb \advance\pgf@yc by-5pt
        % construct main path
        \pgfpathmoveto{\pgfpoint{\pgf@xa}{\pgf@ya}}
        \pgfpathlineto{\pgfpoint{\pgf@xa}{\pgf@yb}}
        \pgfpathlineto{\pgfpoint{\pgf@xc}{\pgf@yb}}
        \pgfpathlineto{\pgfpoint{\pgf@xb}{\pgf@yc}}
        \pgfpathlineto{\pgfpoint{\pgf@xb}{\pgf@ya}}
        \pgfpathclose
        % add little corner
        \pgfpathmoveto{\pgfpoint{\pgf@xc}{\pgf@yb}}
        \pgfpathlineto{\pgfpoint{\pgf@xc}{\pgf@yc}}
        \pgfpathlineto{\pgfpoint{\pgf@xb}{\pgf@yc}}
        \pgfpathlineto{\pgfpoint{\pgf@xc}{\pgf@yc}}
    }
}

\ExplSyntaxOn
% sets the given sequence to the list of paths in \Ginput@path
\cs_new:Nn \set_from_ginput_path_seq:N {
    \seq_clear_new:N #1
    \tl_map_inline:Nn \Ginput@path {
        \tl_if_blank:eF { ##1 } { \seq_put_right:Nn #1 { ##1 } }
    }
}

% Check if file exists in the graphics path
\prg_new_conditional:Npnn \file_if_exist_in_graphicspath:n #1 { p, T, F, TF } {
    \set_from_ginput_path_seq:N \l_file_search_path_seq
    \file_if_exist:nTF { #1 } { \prg_return_true: } { \prg_return_false: }
}

\cs_set_eq:NN \IfFileExistsInGraphicsPathTF \file_if_exist_in_graphicspath:nTF
\cs_set_eq:NN \IfFileExistsInGraphicsPathT \file_if_exist_in_graphicspath:nT
\cs_set_eq:NN \IfFileExistsInGraphicsPathF \file_if_exist_in_graphicspath:nF

% % Check if the extension of the file is in the list of known extensions
% \prg_new_conditional:Npnn \file_is_graphics:n #1 { p, T, F, TF } {
%     \set_from_ginput_path_seq:N \l_file_search_path_seq
%     \tl_clear_new:N \l_file_full_name_tl
%     \file_get_full_name:nN { #1 } \l_file_full_name_tl
%     \file_parse_full_name:nNNN { \l_file_full_name_tl } \l_file_path_tl \l_file_base_tl \l_file_extension_tl
%     \par ext:\tl_to_str:N \l_file_extension_tl \par
%     % \exp_args:Nox \clist_if_in:nnTF {\Gin@extensions} {\l_file_extension_tl} { \prg_return_true: } { \prg_return_false: }
% }

% \cs_set_eq:NN \IfFileIsGraphicsTF \file_is_graphics:nTF
% \cs_set_eq:NN \IfFileIsGraphicsT \file_is_graphics:nT
% \cs_set_eq:NN \IfFileIsGraphicsF \file_is_graphics:nF
\ExplSyntaxOff
\makeatother

% Add a note from the editor to the proof-reader(s)
% #1: the note
\newcommand{\editornote}[1]{\todo[color=TUDa-1a!80]{#1}}
\IfClassLoadedTF{beamer}{
    \newcommand\boldimpact[1]{\centering\huge\vskip\fill\fatsf{\textcolor{.}{#1}\vskip\fill}}
    \newcommand\impact[1]{\centering\huge\vskip\fill\textcolor{.}{#1}\vskip\fill}
    \DeclareDocumentCommand{\urlslide}{om}{
        \begin{frame}[c]
            \slidehead{}
            \begin{figure}
                \centering
                \qrcode[height=3cm]{#2}
                \caption{\IfNoValueF{#1}{#1\\}\url{#2}}
            \end{figure}
        \end{frame}
    }
}{}

% command to change style of single row in tabular
\newcolumntype{$}{>{\global\let\currentrowstyle\relax}}
\newcolumntype{^}{>{\currentrowstyle}}
\newcommand{\rowstyle}[1]{\gdef\currentrowstyle{#1}%
    #1\ignorespaces
}
