% !TeX root = presentation.tex
\documentclass[
    USenglish,
    accentcolor=9c,
    % dark_mode, % uncomment for dark mode
    fontsize= 12pt,
    a4paper,
    aspectratio=169,
    colorback=true,
    fancy_row_colors,
    leqno,
    fleqn,
    boxarc=3pt,
    fleqn,
    main,
    design=2008, % TODO: comment this line, if you want to use the new design
    % shell_escape = false, % Kompatibilität mit sharelatex
]{algoslides}
\RequirePackage{import}
\subimport{common}{packages}
\subimport{common}{macros}
\subimport{common}{style}
\subimport{common}{metadata}
\graphicspath{{pictures/}{pictures/tex/}}
\date{January 01, 2025}
\titlegraphic{\includegraphics[width=\width,height=\height]{example-image-a}}
\begin{document}
    % TODO: replace with actual presentation content
    \maketitle{}

    % eye catching image
    \begin{frame}
        \slidehead{}
        \vspace{-2em}
        \begin{figure}
            \centering
            \includegraphics[height=.6\paperheight]{example-image-b}
            \caption{Eyecatching image, source: \href{https://www.example.com/images/something}{Website Name}}
            \label{fig:paperwork-horror}
        \end{figure}
    \end{frame}
    \section*{Structure}
    \begin{frame}[c]
        \slidehead{}
        \tableofcontents[hideothersubsections]{}
    \end{frame}

    \section{Introduction}

    \subsection{Problem Statement}
    \begin{frame}
        \slidehead{}
        Research Questions
        \begin{itemize}
            \item RQ1
                \begin{itemize}
                    \item RQ1.1
                    \item ...
                \end{itemize}
        \end{itemize}
        some text
    \end{frame}

    \subsection{Motivation}
    \begin{frame}
        \slidehead{}
        \begin{columns}
            \begin{column}{.5\textwidth}
                General motivation
                \begin{itemize}
                    \item blah
                    \item blub
                \end{itemize}
            \end{column}
            \begin{column}{.5\textwidth}
                Personal motivation
                \begin{itemize}
                    \item foo
                    \item bar
                    \item baz
                \end{itemize}
            \end{column}
        \end{columns}
    \end{frame}

    \subsection{Drawbacks of Existing Tools}
    \begin{frame}
        \slidehead{}
        Existing tools have at least one of the following drawbacks:
        \begin{itemize}
            \item mee
            \item moo
            \item maa
        \end{itemize}

        $\Rightarrow$Our platform aims to address these issues.
    \end{frame}

    \section{Methodology}
    \subsection{Architecture}
    \begin{frame}[c]
        \slidehead{}\vspace{-1em}
        \begin{figure}
            \centering
            \includestandalone[scale=.8]{pictures/tex/project-architecture}
            \caption{Project Architecture}
            \label{fig:project-architecture}
        \end{figure}
    \end{frame}
    \subsection{Job Queue}
    \begin{frame}
        \slidehead{}
        \vspace{-2em}
        \begin{figure}
            \centering
            % \includestandalone[height=.6\paperheight]{pictures/tex/operator-job-queue}
            \begin{adjustbox}{totalheight=.6\paperheight}
                \tikzset{
                    archnode/.style={draw,rectangle, minimum size=1cm, text width=1cm, align=center,font=\sffamily},
                    jobnode/.style={archnode, shape=document},
                    runnerscloud/.style={
                        archnode,
                        shape=cloud,
                        cloud puffs=15,
                        text width=3cm,
                        text height=3cm,
                        thick,
                        % split in four parts with dashed lines
                        append after command={%
                            \pgfextra{
                                \draw[dashed, shorten >=.5pt,shorten <=.5pt] (\tikzlastnode.north) -- (\tikzlastnode.south);
                                \draw[dashed, shorten >=.5pt,shorten <=.5pt] (\tikzlastnode.west) -- (\tikzlastnode.east);
                            }
                        }
                    },
                    runner/.style={archnode, very thick,text width=1.5cm, minimum height=1.5cm, font=\small},
                    resultsnode/.style={archnode, text width=2cm, minimum height=1.5cm},
                }
                \def\funnelwidth{1cm}
                \def\runnercenterdist{1mm}
                \def\idlecolor{TUDa-1\IfDarkModeTF{a}{b}}
                \def\runningcolor{TUDa-3\IfDarkModeTF{a}{c}}
                \begin{tikzpicture}[thick,node distance=.5cm]
                    % jobs queue
                    \node[jobnode](Job-1){\textrm{B}\\[.5ex]\faFlask{}\\[.5ex]Job 1};
                    \node[jobnode,right=of Job-1](Job-2){\textrm{A}\\[.5ex]\faFlask{}\\[.5ex]Job 2};
                    \node[jobnode,right=of Job-2](Job-3){\textrm{A}\\[.5ex]\faFlask{}\\[.5ex]Job 3};
                    \node[jobnode,right=of Job-3](Job-4){\textrm{A}\\[.5ex]\faFlask{}\\[.5ex]Job 4};
                    \node[jobnode,right=of Job-4](Job-5){\textrm{C}\\[.5ex]\faFlask{}\\[.5ex]Job 5};
                    \node[right=of Job-5,inner sep=0pt,outer sep=0pt](MoreJobs){\faEllipsisH{}};

                    % \draw[Latex-] ([xshift=-.3cm]Job-1.west) -- node[above]{Queue} ++(-1cm,0cm);

                    \coordinate (PipelineFunnelEnd) at ([yshift=-2cm]$(Job-1)!.2!(MoreJobs)$);

                    % runners cloud
                    \node[runnerscloud,below right=2cm and 3cm of PipelineFunnelEnd](runnersCloud){};
                    \node[above=1ex of runnersCloud,align=left](RunnersLabel){\makebox[1.5em][l]{\faCloud{}}Runners Space};
                    \node[runner,anchor=south east,\idlecolor] at ([shift={(-\runnercenterdist,\runnercenterdist)}]runnersCloud.center){\textrm{A}\\[.5ex]\faServer{}\\[.5ex]Runner 1};
                    \node[runner,anchor=south west,\runningcolor] at ([shift={(\runnercenterdist,\runnercenterdist)}]runnersCloud.center){\textrm{A}\\[.5ex]\faServer{}\\[.5ex]Runner 2};
                    \node<2->[runner,anchor=north east,\idlecolor] at ([shift={(-\runnercenterdist,-\runnercenterdist)}]runnersCloud.center){\textrm{B}\\[.5ex]\faServer{}\\[.5ex]Runner 3};
                    \node<3->[runner,anchor=north west,\idlecolor] at ([shift={(\runnercenterdist,-\runnercenterdist)}]runnersCloud.center){\textrm{A}\\[.5ex]\faServer{}\\[.5ex]Runner 4};

                    % Result
                    \node[resultsnode,right=1.5cm of runnersCloud](Result){\faDatabase{}\\Database};
                    \node[above=1ex of Result](ResultLabel){Results};

                    % pipeline
                    \begin{scope}[on background layer]
                        \draw[thick] ([xshift=-1ex]Job-1.west) --++ (0cm,-1cm) -- ([xshift=-.5\funnelwidth]PipelineFunnelEnd) |- ([yshift=-.5\funnelwidth,xshift=2pt]runnersCloud.west);
                        \draw[thick] ([yshift=-.5\funnelwidth,xshift=-2pt]runnersCloud.east) -- ([yshift=-.5\funnelwidth]Result.west);
                        \draw[thick] ([xshift=1ex]MoreJobs.east) --++ (0cm,-1cm) -- ([xshift=.5\funnelwidth]PipelineFunnelEnd) |- ([yshift=.5\funnelwidth,xshift=2.2pt]runnersCloud.west);
                        \draw[thick] ([yshift=.5\funnelwidth,xshift=-2.2pt]runnersCloud.east) -- ([yshift=.5\funnelwidth]Result.west);
                    \end{scope}
                    \coordinate (PipelineCorner) at (PipelineFunnelEnd |- runnersCloud.west);
                    \coordinate(PipelineVMid) at ($(PipelineFunnelEnd)!.5!(PipelineCorner)$);
                    % arrows
                    \draw[-Latex,dotted] ([yshift=.75cm]PipelineVMid) --++ (0cm,-1.5cm);
                    \draw[-Latex,dotted] ([xshift=-.75cm]$(PipelineCorner)!.5!(runnersCloud.west)$) --++ (1.5cm,0cm);
                    \draw[-Latex,dotted] ([xshift=-.5cm]$(runnersCloud.east)!.5!(Result.west)$) --++ (1cm,0cm);

                    % legend
                    \node[legend] at ([shift={(-1ex,-1ex)}]current bounding box.north east){
                        \textcolor{\idlecolor}{\makebox[1.5em][l]{\faCircle[regular]{}}Idle}\\%
                        \textcolor{\runningcolor}{\makebox[1.5em][l]{\faCircle[regular]{}}Running}
                    };
                \end{tikzpicture}
            \end{adjustbox}
            \caption{Operator Job Queue}
            \label{fig:operator-job-queue}
        \end{figure}
    \end{frame}
    \subsection{Rubric Format}
    \begin{frame}
        \slidehead{}
        \begin{itemize}
            \item JSON-based format
            \item Allows for arbitrarly complex grading schemes
            \item Supports multiple languages
            \item Example on the next slide
        \end{itemize}
    \end{frame}

    \begin{frame}[fragile]
        \slidehead{}
        \vspace{-2em}
        \begin{grayInfoBox}
            \fatsf{Assignment:} Write a method fibRec that calculates the n-th Fibonacci number for any given integer n.
        \end{grayInfoBox}
        \begin{codeBlock}[fontsize=\footnotesize]{minted language=java,title=\codeBlockTitle{Student implementation}}
            package test;
            public class Fibonacci {
                public static int fibRec(final int n) {
                    if (n < 0) {
                        throw new IllegalArgumentException("Parameter n must be greater than or equal to 0, but was " + n);
                    }
                    return n < 2 ? n : fibRec(n - 1) + fibRec(n - 2);
                }
            }
        \end{codeBlock}
    \end{frame}
    \begin{frame}[fragile]
        \slidehead{}
        \vspace{-2em}
        \begin{codeBlock}[fontsize=\footnotesize]{minted language=json,title=\codeBlockTitle{Resulting rubric}}
            {
                "submissionInfo": {},
                "totalPointsMin": 1,
                "totalPointsMax": 1,
                "possiblePointsMax": 2,
                "criteria": [
                    // ...
                ],
                "tests": [
                    // ...
                ]
            }
        \end{codeBlock}
    \end{frame}
    \begin{frame}[fragile]
        \begin{codeBlock}[fontsize=\footnotesize]{minted language=json,title=\codeBlockTitle{Resulting rubric}}
            "criteria": [
                {
                    "name": "H4 | Fibonacci",
                    "archivedPointsMin": 1,
                    "archivedPointsMax": 1,
                    "possiblePointsMin": 0,
                    "possiblePointsMax": 2,
                    "message": "",
                    "relevantTests": [
                        "test1",
                        "test2"
                    ],
                    "children": [
                        // ...
                    ]
                }
            ],
        \end{codeBlock}
    \end{frame}
    \begin{frame}[fragile]
        \begin{codeBlock}[fontsize=\footnotesize]{minted language=json,title=\codeBlockTitle{Resulting rubric}}
            {
                "name": "Die Methode fibRec() gibt die korrekten Fibonacci-Zahlen zurück für negative Eingaben.",
                "archivedPointsMin": 0,
                "archivedPointsMax": 0,
                "possiblePointsMin": 0,
                "possiblePointsMax": 1,
                "message": "<p>Unexpected exception thrown: java.lang.IllegalArgumentException: Parameter n must be greater than or equal to 0, but was -1</p>",
                "relevantTests": [
                    "[engine:junit-jupiter]/[class:h01.FibTest]/[method:testFibRecNegative()]"
                ]
            }
        \end{codeBlock}
    \end{frame}
    \begin{frame}[fragile]
        \begin{codeBlock}[fontsize=\footnotesize]{minted language=json,title=\codeBlockTitle{Resulting rubric}}
            {
                "id": "[engine:junit-jupiter]/[class:h01.FibTest]/[method:testFibRecPositive()]",
                "name": "testFibRecPositive()",
                "type": "TEST",
                "status": "SUCCESSFUL"
            },
        \end{codeBlock}
    \end{frame}
    \begin{frame}[fragile]
        \begin{codeBlock}[fontsize=\footnotesize]{minted language=json,title=\codeBlockTitle{Resulting rubric}}
            {
            "id": "[engine:junit-jupiter]/[class:h01.FibTest]/[method:testFibRecNegative()]",
            "name": "testFibRecNegative()",
            "type": "TEST",
            "status": "FAILED",
            "message": "Unexpected exception thrown: java.lang.IllegalArgumentException: Parameter n must be greater than or equal to 0, but was -1",
            "stackTrace": "org.opentest4j.AssertionFailedError: Unexpected exception thrown: java.lang.IllegalArgumentException: Parameter n must be greater than or equal to 0, but was -1\n\tat org.junit.jupiter.api.AssertionFailureBuilder.build(AssertionFailureBuilder.java:152)\n\tat ... 7 more\n"
        }
        \end{codeBlock}
    \end{frame}
    \begin{frame}
        \begin{figure}[H]
            \includegraphics[height=.6\paperheight]{example-image-a}
            \caption{Minimal Rubric View example}
            \label{fig:minimal-rubric-view}
        \end{figure}
    \end{frame}
    \section{Future Work}
    \begin{frame}
        \slidehead{}
        \begin{itemize}
            \item do
            \item re
            \item mi
            \item fa
            \item so
        \end{itemize}
    \end{frame}

    % \section{References}
    % \begin{frame}
    %     \slidehead{}
    %     \todo{References}
    % \end{frame}

    % \section{Live Demo}
    % \begin{frame}
    %     \slidehead{}
    %     \boldimpact{\textcolor{accentcolor}{Live Demo}}
    % \end{frame}

    \begin{frame}
        \boldimpact{Thank you for your attention!\\\textmd{Do you have any Questions?}}
    \end{frame}

    \section{Appendix}
    \begin{frame}
        \slidehead{}
        \vspace{-2em}
        \begin{figure}
            \centering
            \includestandalone[height=.6\paperheight]{user-groups-kiviat-diagram}
            \caption{Kiviat Diagram of User Groups}
            \label{fig:user-groups-kiviat-diagram}
        \end{figure}
    \end{frame}
    % TODO: Screenshots of the platform
    % \begin{frame}
    %     \slidehead{}
    %     \vspace{-2em}
    %     \begin{figure}
    %         \centering
    %         \includegraphics[height=.6\paperheight]{dashboard-light.png}
    %         \caption{Dashboard}
    %         \label{fig:dashbaord}
    %     \end{figure}
    % \end{frame}
    % \begin{frame}
    %     \slidehead{}
    %     \vspace{-2em}
    %     \begin{figure}
    %         \centering
    %         \includegraphics[height=.6\paperheight]{rubric-light.png}
    %         \caption{Rubric view}
    %         \label{fig:rubric-view}
    %     \end{figure}
    % \end{frame}
    % \begin{frame}
    %     \slidehead{}
    %     \vspace{-2em}
    %     \begin{figure}
    %         \centering
    %         \includegraphics[height=.6\paperheight]{jobs-light.png}
    %         \caption{Jobs view}
    %         \label{fig:jobs-view}
    %     \end{figure}
    % \end{frame}
    % \begin{frame}
    %     \slidehead{}
    %     \vspace{-2em}
    %     \begin{figure}
    %         \centering
    %         \includegraphics[height=.6\paperheight]{assignment-submissions-light.png}
    %         \caption{Assignment submissions view}
    %         \label{fig:assignment-submissions-view}
    %     \end{figure}
    % \end{frame}
\end{document}
